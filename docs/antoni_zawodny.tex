\documentclass[11pt,a4paper]{article}
\usepackage[a4paper, left=1.5cm, right=1.5cm, top=0.5cm, bottom=0.5cm]{geometry}
\usepackage{polski}
\usepackage[utf8]{inputenc}
%\usepackage[T1]{fontenc}
%\usepackage[MeX]{polski}
\usepackage{array}
\usepackage{amsmath}
\usepackage{amsfonts}
\usepackage{amssymb}
\usepackage{bbm}
\usepackage{indentfirst}
\usepackage{xcolor}
\usepackage{listings}

\newcommand{\jpp}[1]{\textit{\color{blue}{#1}}}

\definecolor{codegreen}{rgb}{0,0.6,0}
\definecolor{codegray}{rgb}{0.5,0.5,0.5}
\definecolor{codepurple}{rgb}{0.58,0,0.82}
\definecolor{backcolour}{rgb}{0.95,0.95,0.92}

\lstdefinestyle{mystyle}{
    backgroundcolor=\color{backcolour},   
    commentstyle=\color{codegreen},
    keywordstyle=\color{magenta},
    numberstyle=\tiny\color{codegray},
    stringstyle=\color{codepurple},
    basicstyle=\footnotesize,
    breakatwhitespace=false,         
    breaklines=true,                 
    captionpos=b,                    
    keepspaces=true,                 
    numbers=left,                    
    numbersep=5pt,                  
    showspaces=false,                
    showstringspaces=false,
    showtabs=false,                  
    tabsize=2
}
 
\lstset{emph={ref, string, print},emphstyle={\color{magenta}},style=mystyle}

%opening
\title{JPP language implementation declaration}
\author{Antoni Zawodny}
\date{}
\begin{document}
\maketitle
\section*{Language description}
Long story short - \emph{JPP} language (Just Programming Paralanguage) is a simple imperative language with \emph{C-like} syntax supporting standard arithmeticvariables declarations and definitions, \jpp{if} and \jpp{if-else} statements, \jpp{while} loops with \jpp{break} and \jpp{continue} instructions. Language typing is static and includes only 3 basic types (\jpp{int}, \jpp{bool}, \jpp{string}). Function definitions are supported anywhere in the code (with static binding) as well with arguments passed by value or by reference (with \jpp{ref} keyword). Runtime errors (like division by zero) will be handled as well.

\section*{Grammar}
Included in the EBNF notation in file \emph{grammar.cf}.

\section*{Examples}
All of the following examples will parse correctly and all of the \jpp{print} statements would output \jpp{True} on \emph{stdout}.

\begin{lstlisting}[language=C, caption=Simple expressions]
int funName(int a, int b)
{
    return a + b;
}

int c = funName(21, 21), d = 1, e;

print c == 42;

print c * d == 42;

e = 7

print (2 + 2 * 2) * e == 42;
\end{lstlisting}

\begin{lstlisting}[language=C, caption=Simple loops]
int x = 0, i = 0, j = 0, k = 0;

while (i < 10)
{
    while (j < 10)
    {
        while (k < 10)
        {
            x = x + 1;
            k = k + 1;
        }
        k = 0;
        j = j + 1;
    }
    j = 0;
    i = i + 1;
}

print x == 1000;
\end{lstlisting}

\newpage

\begin{lstlisting}[language=C, caption=Nested definitions of functions]
int incWrapper(int a)
{
    int properInc(int b)
    {
        return b + 1;
    }
    return properInc(a);
}

print incWrapper(41) == 42;
print incWrapper(incWrapper(40)) == 42;
\end{lstlisting}

\begin{lstlisting}[language=C, caption=Recursion]
int factorial(int n)
{
    if (n <= 0)
    {
        return 1;
    }
    else
    {
        return n * factorial(n - 1);
    }
}

print factorial(5) == 120;

print factorial(factorial(3)) == 720;

int dumbFibonacci(int n)
{
    if (n <= 0)
    {
        return 0;
    }
    else if (n == 1)
    {
        return 1;
    }
    else
    {
        return dumbFibonacci(n - 1) + dumbFibonacci(n - 2);
    }
}

print dumbFibonacci(9) == 34;
\end{lstlisting}

\begin{lstlisting}[language=C, caption=Function arguments passed by reference]
int inc(int ref x)
{
    x = x + 1;
    return x;
}

string appendExclamation(string ref s)
{
    s = s + "!";
    return s;
}

int x = 0;
string s = "Hello world";

print inc(x) == 1;
print x == 1;

print s == "Hello world";
print appendExclamation(s) == "Hello world!";
print s == "Hello world!";
\end{lstlisting}

\section*{Grading expectation}
I expect to obtain 27 out of 30 points possible for implementing everything described in this doc and the grammar. According to the description of the interpreter assignment, I would implement every feature up to 20 points plus the type checker (4 points), nested function definitions (2 points) and \jpp{break} and \jpp{continue} instructions (1 point).

\end{document}